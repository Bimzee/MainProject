\chapter{CONCLUSION}\label{chp:chapter5}
\thispagestyle{fancy}
The paper proposes a  methodology to automatically
inject realistic attacks in web applications. This methodology consists of analyzing the web application and generating a set of potential vulnerabilities. Each vulnerability is
then injected and various attacks are mounted over each
one. The attack injection consists of the introduction of realistic vulnerabilities that are afterwards automatically exploited. The success of each attack is automatically assessed
and reported.To understand the feasibility of the method an attack injection tool VAIT is proposed which will try to automate the process with less human effort.
The proposed tool can highlight and overcomes implementation specific issues.The system emphasized the need to match the results of the dynamic and the static analysis of the web applications.The toll will uses Linux, Apache, MySQL and Php for the evaluation purpose.
The VAIT tool can be used as a practical environment that can be used to test
counter measure mechanisms (such as IDS, web application vulnerability scanners, firewalls, etc.), train and evaluate security teams, estimate security measures.